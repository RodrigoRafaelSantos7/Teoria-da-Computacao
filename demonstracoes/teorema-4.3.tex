\subsection{Se $L1$ e $L2$ são regulares, então $L1 \cap L2$ também é regular.}

\begin{spacing}{1.5}
  Como $L_1$ e $L_2$ são regulares, então existem autómatos finitos deterministas
  $(AFD's)$ $M_1 = (S_1, \Sigma, \delta_1, s_1, F_1)$ e $M_2 = (S_2, \Sigma, \delta_2, s_2, F_2)$ completos, que aceitam $L_1$ e $L_2$ respetivamente. Vamos construir um $AFD$ $M = (S, \Sigma, \delta, s, F)$ que aceita $L_1 \cap L_2$.
  Vamos seguir a estratégia em que dado um input $w$, simulamos as computações de $M_1$ e $M_2$ em $w$, lado-a-lado, e aceitamos $w$ se ambas as simulações aceitarem $w$. Para isso, precisamos de saber os estados atuais de $M_1$ e $M_2$ a cada momento da computação de $w$. Definimos então $S = S_1 \times S_2$ em que cada par representa um estado atual possível de $M$. Se $M_1$ está em $q_1$ e $M_2$ está em $q_2$, então o estado atual de $M$ é o tuplo $(q_1, q_2) \in S$ com $q_1 \in S_1$ e $q_2 \in S_2$. Se $M$ está em $(q_1,q_2) \in S$ e lê o simbolo $a \in \Sigma$ então temos que atualizar o estado $q_1$ para $\delta_1(q_1, a)$ e o estado $q_2$ para $\delta_2(q_2, a)$. Assim, definimos a função de transição $\delta$ como $\delta((q_1,q_2), a) = (\delta_1(q_1, a), \delta_2(q_2, a))$. Para o estado inicial de $M$ queremos escolher o par $(s_1,s_2)$ (estados iniciais de $M_1$ e $M_2$ respetivamente), ou seja $s = (s_1,s_2)$. Falta agora definir $F$. Queremos aceitar $w$ qualquer, se ambas as computações em simultâneo aceitarem $w$ (Ou seja terminem num estado final de $M_1$ e num estado final de $M_2$). Assim vamos escolher o conjunto dos pares $(q_1,q_2)$ em que $q_1 \in F_1$ e $q_2 \in F_2$, ou seja $F = F_1 \times F_2$. Assim, o $AFD$ $M = (S, \Sigma, \delta, s, F)$ aceita $L_1 \cap L_2$, mas precisamos de demonstrar que $L(M) = L_1 \cap L_2$. Seja $w \in \Sigma^{*}$ qualquer. A sequência de estados gerada por $w$ em $M_1$ e $M_2$ pode ser descrita como $r_0^{i}, r_1^{i}, \hdots, r_n^{i}$ para $i \in {1,2}$. Se $w \in L_1 \cap L_2$ então por definição temos $r_n^{1} \ \wedge \ r_n^{2}$ pertencentes a $F_1$ e $F_2$ respetivamente. Concluimos então que $\delta(w) = (r_n^{1},r_n^{2}) \in F$. Como $w$ é arbitrário, então $L_1 \cap L_2 \subseteq L(M)$. Vamos ao complementar, se $w \notin L(M)$, então temos $r_n^{1} \notin F_1 \ \wedge \ r_n^{2} \notin F_2$. O que se traduz para $(r_n^{1},r_n^{2}) \notin F$, e portanto $w \notin L(M)$ que leva a $L(M) \subseteq L_1 \cap L_2$. Concluimos então que $L(M) = L_1 \cap L_2$ e portanto $L_1 \cap L_2$ é regular.
\end{spacing}