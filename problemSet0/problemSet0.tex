\documentclass{article}
\usepackage[
    left=1.2in,
    right=1.2in,
    top=0.4in,
    bottom=0.7in,
    paperheight=11in,
    paperwidth=8.5in
]{geometry}

\usepackage{amsthm}
\usepackage{tikz}
\usepackage{amssymb}
\usetikzlibrary{shapes.geometric,fit}

\usepackage{layout}
\title{Teoria da computação \large \\ Problem set 0}
\author{Rodrigo Santos\\
  \small Universidade NOVA de Lisboa\\\\
}
\date{\vspace{-5ex}}

\begin{document}
\maketitle

\subsection*{Exercício 1}
Simplifique os seguintes conjuntos:

\subsubsection*{(a) $(\{1,3,5\} \cup \{3,1\}) \cap \{3,5,7\} $}
$(\{1,3,5\} \cup \{3,1\}) \cap \{3,5,7\} = \{1,3,5\} \cap \{3,5,7\} = \{3,5\}$.

\subsubsection*{(b) $(\{1,2,5\} \setminus \{5,7,9\}) \cup (\{5,7,9\} \setminus \{1,2,5\})$}
$(\{1,2,5\} \setminus \{5,7,9\}) \cup (\{5,7,9\} \setminus \{1,2,5\}) = \{1,2\} \cup \{7,9\} = \{1,2,7,9\}$

\subsubsection*{(c) $\mathbb{Z} \cap [-1,1[ $}
$\mathbb{Z} \cap [-1,1[ \ = \{-1,0\}$

\subsubsection*{(d) $(\mathbb{R} \setminus \mathbb{Z}) \cap [0,1] $}
$(\mathbb{R} \setminus \mathbb{Z}) \cap [0,1] \ = \ ]0,1[$

\subsubsection*{(e) $(\mathbb{Q} \setminus \mathbb{R}) \cup \mathbb{N}$}
$(\mathbb{Q} \setminus \mathbb{R}) \cup \mathbb{N} = \emptyset \cup \mathbb{N} = \mathbb{N}$

\subsection*{Exercício 2}
Qual é a negação lógica das seguintes frases?

\subsubsection*{(a) Está a chover ou a nevar.}
Não está a chover e não está a nevar.

\subsubsection*{(b) Todos os marcianos têm pelo menos um cão.}
Existe pelo menos um marciano que não tem um cão.

\subsubsection*{(c) Existe um ser humano com mais de 3 metros de altura.}
nenhum ser humano tem mais do que 3 metros de altura.

\subsection*{Exercício 3}
Descreva informalmente, mas de forma clara, cada um dos seguintes conjuntos:

\subsubsection*{(a) $\{n \in \mathbb{Z} \ | \ n \geq -100 \wedge n \leq 100\}$}
Todos os números inteiros (positivos e negativos) maiores ou iguais a $-100$ e menores ou iguais $100$.

\subsubsection*{(b) $\{n \in \mathbb{N} \ | \ \exists k (k \in \mathbb{N} \wedge (n = 3k \vee n = 5k))\}$}
Todos os numeros naturais multiplos de 3 ou de 5.

\subsubsection*{(c) $\{(a,b) \in \mathbb{Z} \times \mathbb{Z} \ | \ b = a^2\}$}
Todos os pares de numeros inteiros (positivos e negativos) tais que o elemento $b$ é o quadrado de $a$. Por outras palavras são os pares ordenados em que o segundo elemento é o quadrado do primeiro.

\subsection*{Exercício 4}
Defina os seguintes conjuntos por compreensão:

\subsubsection*{(a) O conjunto dos racionais maiores que 1 e menores que 10.}
$\{n \in \mathbb{Q} \ | \ n > 1 \wedge n < 10\}$

\subsubsection*{(b) O conjunto dos inteiros que são quadrados perfeitos.}
$\{n \in \mathbb{Z} \ | \ n = x^2 \wedge x \in \mathbb{Z}\}$

\subsubsection*{(c) O conjunto dos inteiros que são múltiplos de 10.}
$\{n \in \mathbb{Z} \ | \ n = 10k \wedge k \in \mathbb{N}\}$

\subsubsection*{(d) O conjunto dos inteiros negativos que são múltiplos de 7.}
$\{n \in \mathbb{Z} \ | \ n < 0 \wedge n=7k \wedge k \in \mathbb{N}\}$

\subsubsection*{(e) O conjunto dos racionais que não são inteiros.}
$\{n \in \mathbb{Q} \ | \ n \notin \mathbb{Z}\}$

\subsubsection*{(f) O conjunto dos subconjuntos finitos de N.}
$\{A \ | \ A \subseteq \mathbb{N} \wedge A \in \mathbb{N}^*\}$

\subsection*{Exercício 5}
Se o conjunto $A$ tem $a$ elementos e o conjunto $B$ tem $b$ elementos, quantos elementos tem $A \times B$?

\subsubsection*{Resposta:}
$A \times B$ tem $a * b$ elementos.



\end{document}