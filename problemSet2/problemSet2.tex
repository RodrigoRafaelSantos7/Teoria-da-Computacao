\documentclass{article}
\usepackage[
    left=1.2in,
    right=1.2in,
    top=0.4in,
    bottom=0.7in,
    paperheight=11in,
    paperwidth=8.5in
]{geometry}

\usepackage{amsthm}
\usepackage{tikz}
\usepackage{amssymb}
\usetikzlibrary{shapes.geometric,fit}


\usepackage{layout}
\title{Teoria da computação \large \\ Problem set 2}
\author{Rodrigo Santos\\
  \small Universidade NOVA de Lisboa\\\\
}
\date{\vspace{-5ex}}

\begin{document}
\maketitle
\subsection*{Exercício 1}
Sejam $A$ e $B$ conjuntos quaisquer. Determine, justificando com uma demonstração ou um contra-exemplo, a veracidade das seguintes asserções

\subsubsection*{(a) Se $A$ é contável então $A \cap B$ também é contável.}
Seja $B$ arbitrário $A \cap B \subseteq A$ para qualquer $B$, pela definição de interseção. Portanto $\forall_x \in A \cap B \Rightarrow x \in A$. Temos que todos os elementos de $A \cap B$ estão contidos em $A$, como $A$ é contável, $A \cap B$ é contável.

\subsubsection*{(b) Se $A$ não é contável, então $A \cap B$ também não é contável.}
Seja $A = \mathbb{R}$ que é não contável, \textit{Cantor}. Seja $B = \varnothing$ que é um conjunto contável. Temos então que $A \cap B = \varnothing$ que é conjunto contável. Portanto arranjámos um contra-exemplo, em que $A$ não é contável e $A \cap B$ é contável. Logo a asserção inicial é falsa.

\subsubsection*{(c) Se $A$ é contável, então $A \cup B$ também é contável.}
Seja $A = \varnothing$ que é um conjunto contável. Seja $B = \mathbb{R}$ que é um conjunto não contável, \textit{Cantor}. $A \cup B = \mathbb{R}$ que é um conjunto não contável. Portanto arranjámos um contra-exemplo, em que $A$ é contável e $A \cup B$ não é contável. Logo a asserção inicial é falsa.

\subsubsection*{(d) Se $A$ é contável, então $B \setminus A$ também é contável.}
Seja $A = \varnothing$ que é um conjunto contável. Seja $B = \mathbb{R}$ que é um conjunto não contável, \textit{Cantor}. $B \setminus A = \mathbb{R}$ que é um conjunto não contável. Portanto arranjámos um contra-exemplo, em que $A$ é contável e $B \setminus A$ não é contável. Logo a asserção inicial é falsa.

\subsubsection*{(e)  Se $A$ é contável, então $A^\ast$ também é contável.}
Comecemos por denotar $A^\ast$ como:
\[
  A^\ast = \bigcup_{i \in \mathbb{N}} A^{i} = \epsilon \cup A \cup A^2 \cup \dots \cup A^{i}
\]
Sabemos que $A \times A$ é contável pois é o produto cartesiano entre dois conjuntos contáveis. Logo $A^i = (A \times \cdots \times A) \times A$ é um conjunto contável. Concluimos que se trata da união indexada em $\mathbb{N}$ de conjuntos contáveis, que é contável. Portanto $A^\ast$ é contável se $A$ é contável. Logo a asserção inicial é verdadeira.

\subsection*{Exercício 2}
Determine, justificando, se cada um dos seguintes conjuntos é contável ou não contável
\subsubsection*{(a) O conjunto das funções de \{0, 1\} para \{0, 1\}.}
Comecemos por escrever o conjunto das funções $S$ das funções de \{0, 1\} para \{0, 1\}. Obtemos
\[
  S = \{f: \{0,1\} \mapsto \{0,1\}\} = \bigcup_{n,k \in \{0, 1\}} \{\{(0,k),(1,n)\}\}
\]
Portanto o conjunto $S$ é a união indexada em $\mathbb{N}$ de conjunto do tipo $\{(0,k),(1,n)\}$ com $k,n \in \{0, 1\}$ que são trivialmente finitos, pelo que são contáveis. Logo trata-se da união indexada em $\mathbb{N}$ de conjuntos contáveis que é contável.

\subsubsection*{(b) O conjunto das funções de \{0, 1\} para $\mathbb{N}$.}
Comecemos por escrever o conjunto das funções $S$ das funções de \{0, 1\} para $\mathbb{N}$. Obtemos
\[
  S = \{f: \{0,1\} \mapsto \mathbb{N}\} = \bigcup_{n,k \in \mathbb{N}} \{\{(0,k),(1,n)\}\}
\]
Portanto o conjunto $S$ é a união indexada em $\mathbb{N}$ de conjunto do tipo $\{(0,k),(1,n)\}$ com $k,n \in \mathbb{N}$ que são trivialmente finitos, pelo que são contáveis. Logo trata-se da união indexada em $\mathbb{N}$ de conjuntos contáveis que é contável.

\subsubsection*{(c) O conjunto das funções de $\mathbb{N}$ para $\mathbb{N}$.}

\end{document}

